\documentclass[12pt, letterpaper]{article}
\usepackage{amsmath}
\usepackage{amsfonts}
\usepackage{booktabs}
\usepackage{float}
\usepackage{graphicx}
%\usepackage{multirow}
\PassOptionsToPackage{hyphens}{url}
\usepackage[hidelinks]{hyperref}
\usepackage{indentfirst}
\usepackage[letterpaper, margin=1in]{geometry}
\usepackage{minted}
\usepackage{hyperref}
\usepackage{tablefootnote}

\graphicspath{{./}}

\newcommand\blfootnote[1]{%
  \begingroup
  \renewcommand\thefootnote{}\footnote{#1}%
  \addtocounter{footnote}{-1}%
  \endgroup
}

\begin{document}

\noindent
Erik Marsh

\noindent
erik.i.marsh@gmail.com

\noindent
CS776 -- Assignment 3

\noindent
October 12, 2023

%\tableofcontents

\blfootnote{Code for this assignment can be found at \href{https://github.com/erik-marsh/cs776-as3}{https://github.com/erik-marsh/cs776-as3}.}

\section{Assumptions}

\begin{enumerate}
    \item When the problem specifies a room proportion as ``Any,''
          it will be interpreted as the range $[1, 1.5]$.
    \item Rooms do not have to be positioned in any sort of way.
          The program will simply try to minimize the sum of the costs of each room independently.
    \item Rooms will be defined as a tuple of length, width, and room type.
    \item Room types are as follows: $\{\text{Living, Kitchen, Bath, Hall, Bed1, Bed2, Bed3}\}$.
    \item Chromosomes will be defined as a tuple of rooms (in the exact order of room types as stated above).
    \item Since we can reliably compute maxima and minima for the objective function,
          we can easily define a fitness function that we can maximize.
          The fitness function will be a value in the range $[0, 100]$.
    \item If a room's constraint (length, width, area, or proportion) is not met,
          the cost of the room is set to its maximum possible value.
    \item Proportion can be defined as either $\frac{length}{width}$ or $\frac{width}{length}$.
          Due to this, whichever value is $\geq 1$ is interpreted as the room's proportion.
    \item To save on search time, all chromosomes are initialized with valid (but still random) values.
    \item I use the term ``trial'' to refer to a complete run of the genetic algorithm,
          and the term ``attempt'' to refer to a set of thirty trials. Each trial has a unique random seed.
\end{enumerate}

\vspace{2in}
\pagebreak
\section{Encoding and Decoding Algorithms}

The chromosome layout I was using consists of seven rooms in the following order:
\[\text{Living, Kitchen, Bath, Hall, Bed1, Bed2, Bed3}\]
Each room consists of four floating point values: length, width, x position, and y position\footnote{
    The x and y positions are not actually used for anything.
    I had planned on using them to place the rooms on a grid and enforce constraints on the floor plan of the apartment,
    but that proved to be \textit{very} difficult.
    By the time I realized this, I was already late on this assignment
    and didn't want to re-write my encoding functions.
}.
Every floating point value is encoded as a bitstring of length 10.
To cleanly map onto this bitstring, we restrict floating point values to the range $[0, 102.3]$\footnote{
    I could have used a smaller range (maybe $[0, 25.5]$),
    but if I wanted to place rooms on a grid, I would need a lot more room for the x and y positions.
} with a precision of $0.1$.
The chromosome ends up being a bitstring of length $7 * (4 * 10) = 280$.
The layout of the chromosome can be seen in Figure~\ref{Fig:Bits}.

\begin{figure}[H]
    \centerline{\includegraphics[width=0.75\textwidth]{bit-diagram.png}}
    \caption{The layout of the 280 bit chromosome, each field labeled with its corresponding room value and indices of which bits the field occupies.}\label{Fig:Bits}
\end{figure}

\pagebreak
\section{Genetic Algorithm parameters}

For this assignment, I used a simple genetic algorithm (SGA).
The crossover operator was single-point crossover, and the mutation operator was bit flipping.
Parameter values in \textbf{bold} indicate a parameter that changed from the baseline (a.k.a. attempt 1).

There are a few general things I can say about my results.
While I didn't properly measure the speed of the algorithm,
it ran fast enough that each attempt took no longer than a minute of real time.
Time complexity for each trial of the SGA is on the order of $O(GP)$,
where $G$ is the number of generations, and $P$ is the population size.

In terms of solution reliability and quality,
none of my attempts were able to find the theoretical optimal solution (100 fitness),
but all attempts (except attempt 6) were consistently able to find individuals with over 80 fitness before generation 50.

\subsection{Attempt 1}
\begin{center}
\begin{tabular}{p{0.28\textwidth} l}
    \toprule
    \textbf{Parameter}      & \textbf{Value}    \\ \midrule
    Number of Generations   & 50                \\
    Population Size         & 100               \\
    Crossover Probability   & 0.7               \\
    Mutation Probability    & 0.001             \\
    Chromosome length       & 280               \\ \bottomrule
\end{tabular}
\end{center}

\begin{figure}[H]
    \centerline{\includegraphics[width=0.75\textwidth]{attempt-1/data/fitness-average.png}}
    \caption{Minimum, maximum, and average fitness values for attempt 1.}\label{Fig:Attempt1Fitness}
\end{figure}

I started with these values because they were the suggested values for Assignment 2's SGA
and I needed some sort of baseline to start with.
Fitness appears to improve over each generation, reaching peaks of minimum, maximum, and average fitness by around generation 50.
In terms of reliability and quality, the genetic algorithm very quickly finds an individual with over 80 fitness,
but average fitness never exceeds 80.
The minimum fitness consistently remains low as well, meaning that our generated solutions widely range in quality,
which I interpret as a very unreliable set of solutions.

\subsection{Attempt 2}
\begin{center}
\begin{tabular}{p{0.28\textwidth} l}
    \toprule
    \textbf{Parameter}      & \textbf{Value}    \\ \midrule
    Number of Generations   & \textbf{250}      \\
    Population Size         & 100               \\
    Crossover Probability   & 0.7               \\
    Mutation Probability    & 0.001             \\
    Chromosome length       & 280               \\ \bottomrule
\end{tabular}
\end{center}

\begin{figure}[H]
    \centerline{\includegraphics[width=0.75\textwidth]{attempt-2/data/fitness-average.png}}
    \caption{Minimum, maximum, and average fitness values for attempt 2.}\label{Fig:Attempt2Fitness}
\end{figure}

This time, I adjusted the number of generations to be at least $2 \times$ the population size,
as per the rule of thumb we learned in class.
Unfortunately, this doesn't do very much to the results.
In fact, it appears that the maximum fitness begins to slightly drop off between generations 25 and 50.
However, average fitness climbs slightly higher after generation 50 and appears to plateau near generation 100.
This gives potential solutions with marginally better quality than attempt 1,
but we must still keep in mind the low minimum fitness.
Once again, our algorithm is unreliable.

I think the slightly higher average fitness compared to attempt 1 can be attributed to the increase in number of generations.
This makes sense --- we are giving the algorithm more time to explore new solutions,
and we have a bias toward keeping fitter solutions.
However, the drop-off in maximum fitness is less ideal.
I think this has something to do with the high crossover probability.
Since we are encoding multiple values in a single chromosome,
single point crossover can be quite destructive, yielding solutions that are significantly less fit than the parents.
(If we split the chromosome in the middle of a 10 bit float, the value of that float will vary greatly in the children.)

\subsection{Attempt 3}
\begin{center}
\begin{tabular}{p{0.28\textwidth} l}
    \toprule
    \textbf{Parameter}      & \textbf{Value}    \\ \midrule
    Number of Generations   & \textbf{500}      \\
    Population Size         & 100               \\
    Crossover Probability   & 0.7               \\
    Mutation Probability    & 0.001             \\ 
    Chromosome length       & 280               \\ \bottomrule
\end{tabular}
\end{center}

\begin{figure}[H]
    \centerline{\includegraphics[width=0.75\textwidth]{attempt-3/data/fitness-average.png}}
    \caption{Minimum, maximum, and average fitness values for attempt 3.}\label{Fig:Attempt3Fitness}
\end{figure}

I wanted to see how a progressively higher number of generations would affect fitness in the long term,
hence the increase to 500 generations.
This also does not do very much to solution reliability and quality.
We see very similar results to that of attempt 2.
However, there is an ever more significant drop-off in maximum fitness near generation 100.
Most of what I said about attempt 2 applies here as well.
It seems that progressively increasing the time spent iterating a population is not a very good solution to solving this problem.

\subsection{Attempt 4}
\begin{center}
\begin{tabular}{p{0.28\textwidth} l}
    \toprule
    \textbf{Parameter}      & \textbf{Value}    \\ \midrule
    Number of Generations   & \textbf{750}      \\
    Population Size         & \textbf{300}      \\
    Crossover Probability   & 0.7               \\
    Mutation Probability    & 0.001             \\ 
    Chromosome length       & 280               \\ \bottomrule
\end{tabular}
\end{center}

\begin{figure}[H]
    \centerline{\includegraphics[width=0.75\textwidth]{attempt-4/data/fitness-average.png}}
    \caption{Minimum, maximum, and average fitness values for attempt 4.}\label{Fig:Attempt4Fitness}
\end{figure}

Increasing the number of generations was going nowhere,
so I decided to see what would happen if I increased the population size to 300
(and used an appropriate number of generations as well).

This set of parameters yields the best results yet.
There is a very similar pattern in fitness as described in attempts 2 and 3.
The key difference here is that the average and maximum fitness for each generation
(starting somewhere between generation 50 and 100) is significantly higher than those attempts.
Average fitness appears to float around 80, and maximum fitness floats around 90.
This attempt actually yielded the best individual across all trials, with a fitness of $94.88$.
While peak solution quality seems to have increased, the overall reliability of the algorithm is still not all that great.
The minimum fitness values have stayed in the same range (roughly 40) as the previous two attempts,
meaning that we can still generate a really bad solution in the population.

But this should be good news, right?
Unfortunately, I don't think so.
I think the explanation for the better results here has more to do with the way I initialize my population,
as opposed to the genetic algorithm iterating on the population.
Since I 1) ensure that all of my individuals are initialized to valid values,
and 2) have generated 3 times as many valid individuals than before,
I think the higher maximum and average fitness is due to luck.
I think I just managed to generate better initial solutions because I simply generated more solutions.
This is mostly evident in the maximum fitness pattern.
It looks like a constant was added to that line, as opposed to the algorithm finding higher maxima organically.


\subsection{Attempt 5}
\begin{center}
\begin{tabular}{p{0.28\textwidth} l}
    \toprule
    \textbf{Parameter}      & \textbf{Value}    \\ \midrule
    Number of Generations   & \textbf{250}      \\
    Population Size         & 100               \\
    Crossover Probability   & \textbf{0.3}      \\
    Mutation Probability    & 0.001             \\ 
    Chromosome length       & 280               \\ \bottomrule
\end{tabular}
\end{center}

\begin{figure}[H]
    \centerline{\includegraphics[width=0.75\textwidth]{attempt-5/data/fitness-average.png}}
    \caption{Minimum, maximum, and average fitness values for attempt 5.}\label{Fig:Attempt5Fitness}
\end{figure}

I had the thought that perhaps the chromosome layout was too vulnerable to radical changes from crossover,
and wanted to see what would happen if I made crossover far less likely.
This gives very very similar results to attempts 2 and 3, which I have talked about at length.
It seems that contrary to my initial thoughts, the crossover operator is not that destructive.
In retrospect, this makes sense considering my chromosome layout.
I have two vestigial values in my chromosome that are not used for anything,
meaning that any changes that occur to those values does nothing to the individual at all.
Furthermore, it takes up half of the chromosome, meaning there is a lot of room for nothing to happen.
I think that this also serves as additional support for my theory (see attempt 4) that my genetic algorithm
is good at making lucky initial guesses.

\subsection{Attempt 6}
\begin{center}
\begin{tabular}{p{0.28\textwidth} l}
    \toprule
    \textbf{Parameter}      & \textbf{Value}    \\ \midrule
    Number of Generations   & \textbf{250}      \\
    Population Size         & 100               \\
    Crossover Probability   & 0.7               \\
    Mutation Probability    & \textbf{0.1}      \\ 
    Chromosome length       & 280               \\ \bottomrule
\end{tabular}
\end{center}

\begin{figure}[H]
    \centerline{\includegraphics[width=0.75\textwidth]{attempt-6/data/fitness-average.png}}
    \caption{Minimum, maximum, and average fitness values for attempt 6.}\label{Fig:Attempt6Fitness}
\end{figure}

Finally, I wanted to see what would happen if I massively increased the mutation probability.
After all, crossover doesn't seem to be doing very much to my chromosomes.
This was a mistake, plain and simple.
Minimum, maximum, and average fitness plummet to a little less than 30 in a matter of a few generations.
The quality of solutions here are very poor, but at least this set of parameters is \textit{reliably} giving us poor results.
Additionally, this seems to give even more merit to my ``initial lucky guess'' theory,
as fitness never reaches the same level as the initial generation.

\section{Final Results}

I decided that the best set of results were generated from attempt 4.
The fittest individual that was found across all trials and generations for this attempt
had a fitness of $94.88$ (objective value $663.87$).
As stated in my summary of attempt 4, this set of parameters yields solutions of very high peak quality,
but the overall populations are not reliably high quality.

\begin{center}
\begin{tabular}{p{0.28\textwidth} l}
    \toprule
    \textbf{Parameter}      & \textbf{Value}    \\ \midrule
    Number of Generations   & 750               \\
    Population Size         & 300               \\
    Crossover Probability   & 0.7               \\
    Mutation Probability    & 0.001             \\ 
    Chromosome length       & 280               \\ \bottomrule
\end{tabular}
\end{center}

\begin{figure}[H]
    \centerline{\includegraphics[width=0.75\textwidth]{attempt-4/data/fitness-average.png}}
    \caption{Minimum, maximum, and average fitness values for attempt 4.}\label{Fig:FinalFitness}
\end{figure}

\begin{figure}[H]
    \centerline{\includegraphics[width=0.75\textwidth]{attempt-4/data/objective-average.png}}
    \caption{Minimum, maximum, and average objective function values for attempt 4.}\label{Fig:FinalObjective}
\end{figure}

\section{Best Phenotype}

The phenotype of the fittest individual in attempt 4 is as follows:

\begin{table}[H]
\begin{center}
\begin{tabular}{l l l}
    \toprule
    \textbf{Room Type}      & \textbf{Length}   & \textbf{Width}   \\ \midrule
    Living                  & 9.1               & 13.6             \\
    Kitchen                 & 6.5               & 7.7              \\
    Bath\tablefootnote{Yes, this is an invalid room configuration. This is explained later.}
                            & 9.5               & 26.5             \\
    Hall                    & 5.5               & 3.8              \\
    Bed1                    & 10.0              & 15.0             \\
    Bed2                    & 9.0               & 13.5             \\
    Bed3                    & 8.2               & 12.3             \\ \bottomrule
\end{tabular}
\end{center}
\end{table}

\begin{figure}[H]
    \centerline{\includegraphics[width=0.75\textwidth]{attempt-4/data/best-overall.png}}
    \caption{Visualization of the phenotype for the most fit individual of attempt 4.}\label{Fig:FinalPhenotype}
\end{figure}

Figure~\ref{Fig:FinalPhenotype} warrants some explanation.
The horizontal axis is length, and the vertical axis is width.
The rooms with solid, bright colors are valid,
and the rooms with checkered gray colors are invalid.
The white squares have side lengths of 20.0 units (as a visual aid).
Left-to-right, the top row is Living, Kitchen, Bath, Hall,
and the bottom row is Bed1, Bed2, and Bed3.

Note how the Bath extends far beyond its bounds.
This is due to an oversight in how I calculate room cost in my implementation.
Since the Bath has a constant area,
punishing invalid states by using the Bath's maximum area as the objective function value doesn't work.
Hence, all bathrooms have the same cost.

%\section{Other Interesting Phenotypes}
%nothing in particular

\end{document}